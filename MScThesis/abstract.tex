%% The following is a directive for TeXShop to indicate the main file
%%!TEX root = Thesis_Driver.tex

\chapter{Abstract}
In this master's thesis, I implement a Cooperative Magnetic Inversion (CMI) algorithm for the 3-D modeling of magnetic rocks at depth.
While in most cases it is assumed that the magnetic response is purely induced, certain rocks have the ability to retain a permanent magnetic moment in any orientations, also known as remanent magnetization.
The effect of remanence has long been recognized as an obstacle for the geological interpretation and modeling of magnetic data.
My objective is to improve current magnetic inversion methods to recover simpler and better defined magnetization models.

The CMI algorithm brings together three inversion techniques previously introduced in the literature.
First, magnetic data are inverted for an equivalent-source layer, which is used to derive magnetic amplitude data.
Next, amplitude data are inverted for an effective susceptibility model, providing information about the geometry and distribution of magnetized objects.
Finally, the effective susceptibility model is used to constrain the Magnetic Vector Inversion (MVI), recovering the orientation and magnitude of magnetization.
All three algorithms are formulated as regularized least-squares problems solved by the Gauss-Newton method.

I further constrain the solution by imposing sparsity constraints on the model and model gradients via an approximated $l_p$-norm penalty function. I elaborate a Scaled Iterative Re-weighted Least-Squares (S-IRLS) method, allowing for a stable and robust convergence of the algorithm while combining different $l_p$-norms on the range $0 \leq p \leq 2$.
The goal is to reduce the complexity of magnetization models while also imposing geometrical constraints on the solution.

As a final test, I implement the CMI algorithm on an airborne magnetic survey over the Ekaty Property, Northwest Territories.
I formulate a tiled inversion scheme in order to reduce the computational cost and increase the level of parallelization.
The final merged magnetization model provides insights into the distribution of dyke swarms and kimberlite pipes.
Following the regional inversion, I focus my analysis on sixteen known kimberlite pipes.
Magnetization vector inclinations are compared to the expected polarity of rocks inferred from radiometric dating. 

% Consider placing version information if you circulate multiple drafts
%\vfill
%\begin{center}
%\begin{sf}
%\fbox{Revision: \today}
%\end{sf}
%\end{center}
