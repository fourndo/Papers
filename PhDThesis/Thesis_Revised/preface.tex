%% The following is a directive for TeXShop to indicate the main file
%%!TEX root = Thesis_Driver.tex

\chapter{Preface}
The research presented in this thesis was completed by myself, under the guidance of my supervisor Professor Oldenburg.
The algorithms and ideas brought forward resulted in multiple articles, either published or in revision. 
  
The mixed norm algorithm presented in Chapter~\ref{Chapter3} and part of the learning algorithm presented in Chapter~\ref{Chapter6} resulted in the article \emph{``Inversion using spatially variable mixed $\ell_p$-norm''} published in \emph{Geophysical Journal International} \cite[]{FournierDWO2019}. I am the primary author of this article with revisions from Dr. Oldenburg. The content of the article had previously been presented at a conference \cite[]{FournierDavis2016a}.
The same algorithm has been used by other researchers and resulted in three research papers to which I am also co-author \cite[]{Miller2017, Abedi2018, Abedi2018a}.

Improvements to the magnetic vector inversion in Chapter~\ref{Chapter4} will be featured in the accepted paper \emph{``Sparse magnetic vector inversion in spherical coordinates: Application to the Kevitsa Ni-Cu-PGE magnetic anomaly, Finland''} submitted to \emph{Geophysics} \cite[]{FournierDWO2019}. The article was written by myself under the guidance of Dr. Oldenburg.  I have also co-authored a research paper investigating the geothermal resources at Mount Baker \cite[]{Schermerhorn2017}.
Using the same methodology, I collaborated in the implementation of sparse vector inversion applied to self-potential problems to map hydrothermal circulation at Mount Tongariro, New Zealand \cite[]{Miller2018}. 

The material presented in Chapter~\ref{Chapter5} and part of Chapter~\ref{Chapter7} is currently in preparation for publication. The article entitled \emph{``Sparse rotated objective function for stratigraphic constraints: Application to the Kevitsa Ni-Cu gravity anomaly, Finland''} will be submitted within a few weeks.

All programming work done in this thesis builds upon the open-source \texttt{SimPEG} library as well as multiple packages from the Python ecosystem. Accreditation to open-source algorithms have been made wherever necessary.